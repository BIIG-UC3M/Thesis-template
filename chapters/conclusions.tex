\chapter*{Conclusions}
\addcontentsline{toc}{chapter}{Conclusions}
\markboth{Conclusions}{Conclusions} 


The work presented in this thesis contributes to the comprehension of the propagation regime of light in optical imaging in the \gls{nir} and more in particular in the \gls{nir2} spectral window. The applications of this study are focused on \gls{opt} and \gls{lsfm} since these two techniques have become very popular among the research community and the recent literature shows a strong interest in this spectral windows. Measurements in the \gls{nir} demonstrated to fall in the yet uncharacterized ballistic to diffusion transition regime of light and thus this research covers the development of novel methods to model light propagation, the assessment of existing microscopy techniques in a variety of scattering media and the exploration of new acquisition methods for mesoscopic \gls{opt} to overcome the new challenges that imaging in the \gls{nir2} window and sets. It is important to outline that the findings of this work apply also for any low scattering media where the effects of dispersion cannot be neglected.

The results presented in the thesis aim to predict the advantages of the use of \gls{nir} light in imaging systems from the theoretical framework to real applications. The third chapter proposes a new method to study light propagation in low scattering media that takes into account the physical quantity measured by a detector in a real imaging system. The work introduces a new \gls{mc} simulation method to estimate the flux density of a point source and a collimated pencil beam in an infinite medium evaluated in a set of parallel detection surfaces with high spatial radial resolution. After validating the simulator in highly diffusive media, the tool was used to compare the results predicted by the \gls{da} against the proposed \gls{mc} method, which is known to have excelent agreement with the solution of the \gls{rte}. The simulations were performed for scattering values similar to those reported by the literature for \gls{nir} light to compare the validity of the \gls{da} in this window. The results showed that below a \gls{ltr} the diffusion model fails since the contribution of the reduced intensity is larger than expected in this regime. These findings could be observed also for the collimated beam, demonstrating that the point source approximation is accurate to model directional light beyond a \gls{ltr}. Moreover, as this value in the \gls{nir} can be of up to one order of magnitude higher than in the visible, we concluded that the use of \gls{nir} light for imaging increases the depth of penetration according to the \gls{ltr} distance. This results were calculated for sources placed in an infinite medium however, whether this findings have also validity in current microscopy techniques is an interesting question addressed in chapter \ref{chap:mcspim}.

The increase of the computation power achieved during the last decade allowed to significantly increase the complexity of \gls{mc} simulations. In chapter \ref{chap:mcspim}  we present the development of a \gls{mclsfm} simulator capable of mimicking the entire photon flow of a \gls{ls} microscope to evaluate the performance of this technique with light from the \gls{nir} window. The software was developed with the combination of a modified version of a well validated MC package \cite{Fang2009} with a novel algorithm to focus the detected fluorescence and compute the optical sectioning according to the position of the \gls{ls} in the volume. In order to reduce the computation times, the whole workflow was programmed in the \gls{gpu} and the use of hardware resources vastly optimized. Moreover, several improvements were also implemented to overcome voxel size limitations while recreating very thin \gls{ls}s propagating through large volumes. The images generated by the simulator showed to be realistic, mimicking the loss of transaxial resolution due to the broadening of the light sheet and of axial towards deep z planes due to the scattered fluorescence. This tool was used to simulate a distribution of fluorophores in a large volume with optical properties of tissues in several spectral windows to assess the depth of penetration of \gls{ls} microscopy. The results showed that \gls{lsfm} is expected to penetrate at least a distance of one \gls{ltr} in scattering media, which demonstrates that the predictions of the theoretical framework can be translated into this optical imaging modality. 

The \gls{lsfm} simulator presented in this work was developed to evaluate the effects of scattering in this technique, however we realized that its use can be extended to also simulate other microscopy techniques or even new acquisition methods. Chapter \ref{chap:opt_spim} showcases the use of this tool to validate SPOT, a novel technique in which projections from different views are acquired through the integration of the stack of z planes of a \gls{lsfm} volume. The simulation tool was used to compare the results from this method against traditional \gls{lsfm}, exhibiting isotropic voxel size and a more stable resolution at deep z planes for SPOT. Moreover, the inhomogeneities of the illumination due to the attenuation of the \gls{ls} where suppressed with this new technique. 

The results of chapter \ref{chap:opt_spim} demonstrate the high potential of the tool developed in chapter \ref{chap:mcspim} beyond its initial purpose and introduces a new hybrid imaging model that joins the advantages of \gls{lsfm} and \gls{opt} microscopy techniques into a single modality.

The \gls{mclsfm} simulation software is available for download at \url{https://github.com/amarcosv/MC-LSFM.git}. The repository includes all the source codes, compiled binaries and user manual.

After the large amount of computational work carried out in the previous sections of this thesis, we introduce an experimental proof of concept version of a transmission \gls{opt} system in the \gls{nir2}. The classical illumination method in \gls{opt} seemed to be inappropriate for the optical properties of this window due to the increase in the absorption coefficient of water. This issue was solved with the development of two acquisition methods. The first uses polarized light to suppress the illumination of the background the second uses a scanning galvomirror to illuminate only the region of interest of the \gls{fov}. Both approaches allowed to reconstruct the samples, demonstrating that \gls{nir2} light can reveal structures from highly opaque samples. 

