\chapter*{Motivation and Objectives}
\addcontentsline{toc}{chapter}{Motivation and Objectives}
\markboth{Motivation and objectives}{Motivation and objectives} % Force the right header

Optical instrumentation started its journey when ancient civilizations invented the first rudimentary lenses and derived the geometrical optics laws. Centuries later, the sophistication of the fabrication methods allowed to build more complex apparatus that opened the door to observe a whole new world from the smallest molecules to the remotest galaxies. However, it was just a few decades ago that the invention of technologies to store images revolutionized the way those observations contribute to the progress of science. Light microscopy nowadays has turned inconceivable without all these instruments but, despite the giant achievements of the last years, the physics of light still limit its capability to image in thick biological samples.


The first part of the thesis consists of a comprehensive review of the theoretical background and the state of the art of optical tomographic imaging in low scattering media, serving as the starting point for the research work that will be presented in the rest of the thesis. Then we propose a set of computational tools to investigate the propagation of light from infinite media to microscopy applications. The last section presents the first results of an experimental setup that aims to validate some of the results from previous sections.

Chapter~\ref{chap:theory} describes the optical properties of biological tissues along the spectral windows for imaging and the physical models of light propagation in scattering media. Moreover, this section details the consequences of imaging in diffusive media in terms of depth of penetration and resolution loss and discusses the advantages of the use of \gls{nir} light.

Chapter~\ref{chap:review} reviews the most recent advances in \gls{opt} and \gls{lsfm}. These two techniques have quickly evolved during the last two decades and have a vast variety of variants and acquisition methods. 

Chapter~\ref{chap:flux} proposes a new simulation approach to model the light measured by a detector in scattering media. The software is validated with a well defined diffusive case and simulations are performed for the optical properties that can be found in tissues in the \gls{nir2} window.

Chapter~\ref{chap:mcspim} covers the development of a simulation environment for \gls{lsfm}, presenting its potential as a tool to assess the performance of this technique in the \gls{nir}.

In chapter~\ref{chap:opt_spim}, the simulation tool of the previous chapter is extended to other imaging modalities, demonstrating that this tool can be used to develop and validate new imaging methods.

Chapter~\ref{chap:opt_nir2} presents the experimental results from the development of a transmission \gls{opt} system for the \gls{nir2} window, describing the new challeneges that the optical properties of tissues set to the classical illumination approach.